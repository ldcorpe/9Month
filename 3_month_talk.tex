% !TEX TS-program = pdflatex
% !TEX encoding = UTF-8 Unicode
\documentclass{beamer}
\mode<presentation>
{
  \usetheme{Warsaw}
}

\usepackage[english]{babel}
\usepackage[utf8]{inputenc}
\usepackage{times}
\usepackage[T1]{fontenc}
\usepackage{graphicx}
 \usepackage{sidecap}
 \usepackage{xcolor}
\useoutertheme{infolines}

\title[$H \rightarrow \gamma \gamma$ at CMS]
{Analysis of Higgs bosons decaying to two photons at CMS}

%\subtitle
%{Include Only If Paper Has a Subtitle}

\author[L. D. Corpe] % (optional, use only with lots of authors)
{L. D.~Corpe}


\institute[Imperial College London] % (optional, but mostly needed)
{
  Imperial College London\\
  Blackett Laboratory\\
  London}


\date[2014] % (optional, should be abbreviation of conference name)
{PG Initial Report Talks, 2014}

\subject{Particle Physics}



%\AtBeginSubsection[]
%{
  %\begin{frame}<beamer>{Outline}
    %\tableofcontents[currentsection,currentsubsection]
  %\end{frame}
%}

%%%%%DOCUMENT START%%%%%%%%%%
%%%%%%TITLE a TOC%%%%%%%%%%

\begin{document}

\begin{frame}
  \titlepage
\end{frame}

\begin{frame}{Outline}
  \tableofcontents
\end{frame}

%%%%%%%%%%INTRODUCTION%%%%%%%%

\section{Introduction}


%--------------------SM Higgs intro-----------------
\subsection{The SM Higgs Boson}
%____________slide 1______________________
\begin{frame}{The Standard Model Higgs boson}{History}

\includegraphics[width=0.4\textwidth]{"Sakurai"} \quad
\includegraphics[width=0.177\textwidth]{"Higgs"} \quad
\mbox{\parbox[b][4em][b]{0.3\textwidth}{\tiny The authors of the ``1964 PRL symmetry breaking papers'' won the Sakurai Prize in 2010. 
 \vspace{5px}
 Higgs and Englert won the Nobel prize in 2013.
  \vspace{5px}
  Left: Kibble, Guralnik, Hagen, Englert, Brout. Right: Higgs\\ \\} }

{\footnotesize The Higgs mechanism was independently formulated various theorists in 1964. Explains mass of $W^{\pm}$ and $Z$ bosons via symmetry breaking in the electroweak interaction. Crucially, gauge invariance is conserved. Main properties of SM Higgs:
  \begin{itemize}
\pause
  \item 
  Massive and observable. Now known to be $\sim 125$ GeV.
 \pause 
  \item
  Couples to particles proportional to their mass.
\pause
 \item
 Only one Higgs boson in SM, while other BSM theories predict more.

  \end{itemize}
  }
\end{frame}
%____________slide 2______________________

\begin{frame}{The Standard Model Higgs boson}{Production and decay at LHC}
\begin{itemize}
\item 
{\tiny $H$ couples to particles $\propto m$, so main production modes at LHC are:}
\begin{center}
\includegraphics[width=0.8\textwidth]{"productions"}
\item {\tiny a) $gg$ fusion via $t$ loop, b) Vector Boson Fusion (VBF), c) Assoc. $Z,W$ production, d) Assoc. $t\bar{t}$ production.}
\end{center}
\pause
\item 
{\tiny By the same token, it decays mostly to heavy particles:}
\begin{center}
\includegraphics[width=0.8\textwidth]{"decays"}
\item {\tiny Decay to 1) $b \bar{b}$ pair, 2) vector boson pair, 3) $\tau^+, \tau^-$, 4) two photons via $t$ loop (can also be $W$).}
\end{center}
\item {\tiny $t, \bar{t}$ pair kinematically forbidden. $H \rightarrow \gamma \gamma $ is rare, but one of the most sensitive channels at the LHC.}

\end{itemize}
\end{frame}

%--------------------LHC, CMS , ECAL---------------------------

\subsection{LHC, CMS and ECAL}

%____________slide 1______________________
\begin{frame}{LHC, CMS and the ECAL}{The Large Hadron Collider}
\begin{center}
\includegraphics[width=0.55\textwidth]{"LHC"}
\end{center}
  \begin{itemize}
\item {\footnotesize The LHC is a 27km circumference synchrotron at CERN, as explained by my colleagues.}
  \end{itemize}
\end{frame}
%____________slide 2______________________
\begin{frame}{LHC, CMS and the ECAL}{The Compact Muon Solenoid}
\begin{center}
\includegraphics[width=0.85\textwidth]{"CMSLayout"}
\end{center}
  \begin{itemize}
\item {\footnotesize Explained already by previous talks - I will focus on ECAL layer.\\}
  \end{itemize}
\end{frame}
%____________slide 3______________________
\begin{frame}{LHC, CMS and the ECAL}{The Electromagnetic Calorimeter}
\begin{center}
\includegraphics[width=0.65\textwidth]{"EcalView"}

{\tiny A view of the inside of the ECAL Barrel.\\}
\end{center}
\end{frame}
%____________slide 3______________________
\begin{frame}{LHC, CMS and the ECAL}{The Electromagnetic Calorimeter}

{\footnotesize The CMS ECAL is composed of an array of PbWO$_4$ crystals. \\ The Crystals are offset by an angle of 3$^o$ from the vertex to avoid particles going through the gaps in between.}
 \includegraphics[width=0.5\textwidth]{"ECAL"}
 \mbox{\parbox[b][4em][b]{0.45\textwidth}{
 \begin{itemize}
 \item \fcolorbox{black}{white}{\tiny 62,100 crystals in 36 supermodules  }
 \item \fcolorbox{white}{yellow}{\tiny Supermodule containing 4 modules  }
 \item \fcolorbox{white}{teal}{\tiny Endcaps containing 14,648 crystals }
 \item \fcolorbox{white}{pink}{\tiny Preshower for $\pi^0$ ID.}
 \vspace{1cm}

\end{itemize}}}
\end{frame}


%%%%%%%%%%H TO GAMMA GAMMA %%%%%%%%

\section{$H \rightarrow \gamma\gamma$}

\subsection{Photon and Vertex Identification }

%\begin{frame}{$H \rightarrow \gamma \gamma$ Analysis Strategy}
%\begin{itemize}
%\item The $H \rightarrow \gamma \gamma$ analysis relies on the fact that a small but non negligible fraction of $H$s will decay to two photons via a $t,W^{\pm}$ or $Z$ loop.
%\pause
%\item Although there is a large background from QCD processes and “non-prompt'' photons, the high resolution from the ECAL allows the reconstruction of a narrow peak above the background.
%\pause
%\item Main analysis relies heavily on boosted decision trees (BDTs)
%\end{itemize}
%\end{frame}



\begin{frame}{Photon and Vertex Identification}{Initial event selection and mass reconstruction}
\begin{itemize}
\item {\footnotesize Initial trigger identifies $\gamma$s from ECAL isolation and shower shape. A loose $E_T$ cut is also made, although a tougher one is imposed later to find Higgs decay candidates.}
\pause
\item { \footnotesize Higgs mass reconstructed using the following formula (simple 4-momentum conservation): }
$$ m_H=m_{\gamma\gamma}=\sqrt{2E_{\gamma1}E_{\gamma2}(1-\cos\alpha)} $$

\includegraphics[width=0.45\textwidth]{"Diagram"}
\mbox{\parbox[b][4em][b]{0.5\textwidth}
{\pause \item \footnotesize $E_{\gamma1},E_{\gamma 2}$ dominate mass resolution if primary interaction vertex is identified. Thankfully ECAL has excellent E resolution.
\item However, diphoton interaction vertex must be correctly identified to measure $\alpha$. \vspace{10px}}}
%\pause \item \textbf{The better the vertex reconstruction, the better the angle resolution!} \vspace{0px}}}
\end{itemize}

\end{frame}

\begin{frame}{Photon and Vertex Identification}{Vertex ID and Pair Conversion}
\begin{itemize}
\item If the diphoton vertex is reconstructed with 10mm of true position then the mass resolution is dominated by the $E_{\gamma}$ resolution.
\pause \item A BDT is used to identify vertex using kinematic properties as inputs. Also makes use of extra information from tracker if $\gamma$ has converted to $e^+ e^-$.
\pause \item We can tell if $\gamma$ has converted using $R_9 \equiv  \frac{E_{3x3}}{E_{SuperCluster}} < 0.94$.
\includegraphics[width=0.4\textwidth]{"R9"} \hspace{5px}
\mbox{\parbox[b][4em][b]{0.5\textwidth}
{\footnotesize \item If $\gamma$ hits, most of the energy is deposited within $3x3$ array, so $E_{3x3} \simeq E_{5x5}$ so $R_9 \simeq 1$ 
\item If $\gamma$ has converted, less energy will be focused within $3x3$, so $R_9 < 1$ \vspace{30px}  }}
\end{itemize}
\end{frame}

\begin{frame}{Photon and Vertex Identification}{Higgs decay candidates}
\begin{itemize}
\item Not all diphoton events are of interest! We only want those which are Higgs decay candidates.
 \item Higgs decay photons should be highly energetic, so impose:
$$ E^T_{\gamma1} > \frac{m_{\gamma\gamma}}{3} \text{ and } E^T_{\gamma2} >\frac{m_{\gamma\gamma}}{4}$$
 \item A further BDT is used to remove ``non-prompt'' photons and particles misidentified as photons, as they are of no interest.
%\includegraphics[width=0.4\textwidth]{"R9"}
\end{itemize}
\end{frame}

\subsection{Event Categorisation and Analysis}

\begin{frame}{Event Categorisation and Analysis}
\begin{itemize}
\item Events are segmented into categories based on expected signal to background ratio and mass resolution using a BDT.
\item This increases the overall sensitivity of the analysis.
\item The background is modelled using data rather than Monte Carlo.
\vspace{5px}
\begin{itemize}
\pause \item For each category, the $m_{\gamma\gamma}$ distribution is fitted. Candidate fitting functions include exponentials, power laws, Bernsteins (polynomials) and Laurent series. 
\pause \item Fitting function chosen based on bias minimisation. Bias is negligible for all categories when Bersteins of order 3-5 (depending on category) are used.
\end{itemize}

\end{itemize}
\end{frame}

\begin{frame}{Analysis}
\begin{itemize}{ \footnotesize
\item The mass distribution is plotted for each category and compared to the background prediction. The categories are then combined to form the global analysis.

\item In 2012, this method yielded an observed local significance of $4\sigma$ at $\sim 125$ GeV. Combined with other analyses, the total local significance was $5\sigma$, allowing a discovery to be claimed by CMS (and ATLAS).}
 \includegraphics[width=0.455\textwidth]{"Plot"}
\includegraphics[width=0.4\textwidth]{"PValue"}
\end{itemize}
\end{frame}
\section{Outlook and future work}

\begin{frame}{Outlook}

  \begin{itemize}
  \item A final legacy paper using all data from run 1 is being prepared, currently under approval.
 \pause \item In 2015, the LHC will re-start collisions, hopefully ramping up towards design value of $14$ TeV.
\pause  \item This will allow more precise measurements of couplings, differential cross-sections and $J^P$.
\pause  \item Any deviation from the SM could yield insight on nature or existence of BSM physics, e.g. a 2nd Higgs, or increased cross-section due to heavy BSM particles.
\pause  \item For first time in history, we have found what we believe to be a scalar fundamental particle. Implications unclear.
 \pause \item Although we have found the $H$, further studies are not only desirable but imperative.
 
  \end{itemize}
 
\end{frame}
\begin{frame}{Questions}
\begin{center}
{\Large Thanks for listening! 

Questions?}
\end{center}
\end{frame}

\end{document}


